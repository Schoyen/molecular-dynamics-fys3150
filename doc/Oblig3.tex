\documentclass[11pt, a4paper]{article}

\usepackage[T1]{fontenc}
\usepackage[utf8]{inputenc}
\usepackage{amsmath, amssymb, amsfonts, amsthm}
\usepackage{bbm}
\usepackage{graphicx}
\usepackage{verbatim}
\usepackage{caption}
\usepackage{subcaption}
\usepackage{subfig}
\usepackage{float}

\theoremstyle{definition}
\newtheorem*{definition}{Teorem}

\newcommand{\vb}{\mathbf}

\renewcommand{\contentsname}{Innhald}

\renewcommand{\abstractname}{Samandrag}

\begin{document}

\begin{titlepage}

  \title{\normalsize FYS3150 Computational Physics\\
  \vspace{10mm}
  \huge Oblig 3\\
  \vspace{10mm}
  \normalsize{\bf Molekylær dynamikk.}}

  \author{Fyll inn kandidatnummer.}

\end{titlepage}

\maketitle

\begin{abstract}
  Me må ha ei trippel-løkke for alle indeksane til ei celle og ei indre trippel løkke for alle nabocellene ((i-1, i, i+1), (j-1, j, j+1), (k-1, k, k+1)). 
  Hugs å sjekke med boundary conditions.
  Når me konstruerar kvar celle vil me lagre dei ved indeks
  \begin{align*}
    void \ index(int i, int j, int k) \{ \\
      posisjon = i * n_y * n_z + j * n_z * k; \\
    \}
  \end{align*}
  kor $n_x$, $n_y$ og $n_z$ er storleiken på kvar celle i $x$, $y$ og $z$-retning.
  Storleiken bestemmer me ved
  \begin{align*}
    int \ n_x = \left( \frac{L_x}{r_{cut}} \right),
  \end{align*}
  kor $L_x$ er storleiken på systemet i $x$-retning.
  Metoden for å legge til eit atom tek imot ein atom peikar og sjekkar om atomet ligg i cella.
  \begin{align*}
    void \ addAtom(Atom *atom) \{ \\
      if (atom->position.x() / (systemSize.x()) * cell->position.x()) \{ \\
        add the atom; \\
      \} \\
    \}
  \end{align*}
  Ekstra oppgåve: Prøv å lagre cellene med indeks. Oppdater deretter indeksane på kvart atom i kvar celle slik at me kan bruke N3L for å rekne ut krafta kun ei gong. Viss ikkje 
  må me køyre testar på kvart atom to gonger frå to celler for å hindre at me rekner ut krafta to gonger.
\end{abstract}

\end{document}
