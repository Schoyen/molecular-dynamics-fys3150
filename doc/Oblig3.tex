\documentclass[11pt, a4paper]{article}

\usepackage[T1]{fontenc}
\usepackage[utf8]{inputenc}
\usepackage{amsmath, amssymb, amsfonts, amsthm}
\usepackage{bbm}
\usepackage{graphicx}
\usepackage{verbatim}
\usepackage{caption}
\usepackage{subcaption}
\usepackage{subfig}
\usepackage{float}

\theoremstyle{definition}
\newtheorem*{definition}{Teorem}

\newcommand{\vb}{\mathbf}

\renewcommand{\contentsname}{Innhald}

\renewcommand{\abstractname}{Samandrag}

\begin{document}

\begin{titlepage}
    \begin{center}
        \vspace*{1cm}
        
        \textbf{\huge Molekylær Dynamikk}

        \vspace{0.5cm}
        \textbf{Oblig 3}

        \vspace{1.5cm}
        \textbf{Øyvind Sigmundson Schøyen} \\
        Kandidatnummer: 30

        \vfill

        Avsluttande prosjekt i FYS3150 \\

        \vspace{0.8cm}

        \includegraphics[width=0.4\textwidth]{\string~/Downloads/UiO_Segl_300dpi.png}

        FYS3150 Computational Physics\\
        Universitetet i Oslo\\
        1. desember 2014

    \end{center}
\end{titlepage}

\begin{abstract}
    I dette prosjektet har me tatt for oss modellering av argon på atomært nivå. Me har vore interesserte i å lage eit program som skal lage ein atomstruktur etter eige ynskje.
    I tillegg har me ville sjå på korleis eit slikt system utvikler seg over tid og måle statistiske eigenskapar ved det. For å modellere eit så realistisk resulatat innenfor
    det som er mogleg å køyre på ein pc har me utnytta cellelister for å auke hastigheita på programmet. For krafta har me nytta Lennard Jones potensialet og som integrator
    har me brukt Velocity Verlet-algoritma. Programma våre er objektorienter C++-kode med eit Python rammeverk som skal enklast mogleg køyre programmet vårt for forskjellige 
    parametrar og plotte verdiar. Me nyttar VMD for å visualisere atoma i rommet. All kjeldekode ligg på github.\\ \\
    Rett litt på denne\dots \\ \\
    \texttt{https://github.com/Schoyen/molecular-dynamics-fys3150}
\end{abstract}

\newpage
    \tableofcontents
\newpage

\section*{Introduksjon}
    \addcontentsline{toc}{section}{Introduksjon}
    Oppgåva me er gjevne har gått ut på å modellere eit fysisk system samt bruke objektorientert programmering til å holde ein ryddig, oversiktlig, samt effektiv kode. 
    I byrjinga er me gjeve ein kode som lagar 100 argon atom og gjer dei ein tilfeldig retning og hastighet. Gjeve lang nok tid vil atoma drive vekk. Me vil difor nytte
    ``periodiske randbetingelsar'' for å halde atoma i nærleiken. Grunna hastigheitar gjevne frå Maxwell-Bolzmann distribusjon vil systemet ha ein ikkje-null
    rørslemengde som me vil fjerne. Me vil deretter lage ein krystallstruktur kor me startar simuleringa av atoma. I eit slik lukka system vil total energien vere bevart, 
    men grunna numerisk avrunding er det ikkje alltid dette held mål. Ein stabil alogritme som me vil nytte er ``Velocity Verlet'' som er ein symplektisk integrator. % Forklar dette.
    Me vil no byrje å måle statistiske eigenskapar som energi og temperatur. For å kunne køyre koden for store system vil me derimot utvikle ein kjappar algoritme når 
    me rekner ut krafta mellom atompara. Dette løyser me med cellelister. Til slutt, i fyrste del av prosjektet, legg me til ein termostat som let oss kontrollere 
    temperaturen i systemet. \\ \\
    % Fyll in for del 2 av prosjektet.

\end{document}
